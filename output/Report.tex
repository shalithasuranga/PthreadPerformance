\documentclass[12pt, a4paper]{article}
\usepackage{tikz}
\usepackage{datetime}
\usepackage{pgfplots}
\usepackage{array}
\usepackage{tabu}
\usepackage{tocloft}
\renewcommand{\cftsecleader}{\cftdotfill{\cftdotsep}}
\usepackage{geometry}
\pgfplotsset{compat=1.11}

 \geometry{
 a4paper,
 total={170mm,257mm},
 left=20mm,
 top=20mm,
 }


\font\titlefont=cmr12 at 40pt
\font\metafont=cmr12 at 14pt

\title{\titlefont{Pthread Matrix Multiplication Runtime}}
\author{M.A.S SURANGA (CST140043)}
\date{}

\begin{document}

	\maketitle

	\vspace*{5\baselineskip} 
	\begin{center}
	\metafont{Generated on \today \ \currenttime}
	\end{center}

	\newpage
	\setlength{\cftbeforesecskip}{6pt}
	\tableofcontents

	\newpage
	\section{Introduction}
	\hfill \break

	\textbf{Report Content}

	\hfill \break

	This report contains performance graph for matrix multiplication using pthreads. Runtime is measured for several thread amounts by keeping matrix size as fixed.  

	\hfill \break

	\textbf{Testcases Details}

	\hfill \break


	\begin{tabu} to 1 \textwidth { | X[l] | X[r] | }
		 \hline
		 No. of Threads & \input{meta_threads.dat}  \\
		 \hline
		 No. of iterations & \input{meta_avg.dat}  \\
		 \hline
		 Fixed Matrix Size & \input{meta_fixedmatrix.dat}  \\
		 \hline
	\end{tabu}	
	

	\newpage

	\section{CPU Pthread Program runtime}

	\par
	This program is executed in CPU using \textbf{POSIX threads} and $x,y$ values are extracted from the average runtime of each different threads count.
	\hfill \break

	\begin{tikzpicture}
		\begin{axis}[grid=major, grid style={black!20}, width=\linewidth, xlabel=No. of Threads,ylabel=Runtime(secs.),legend style={at={(0,-0.3)},anchor=south}, ylabel near ticks]
			\addplot[color=red,mark=x] table[ignore chars={(,)},col sep=comma] {datafile0.dat};
			\addlegendentry{$PThreads CPU$}
		\end{axis}
	\end{tikzpicture}

	\begin{itemize}

		\item There is good performance gain when the program is using multiple threads rather than single thread.

		\item When the threads amount is exceeding the actual parallel threads of the CPU runtime plot has random peaks and wells and when threads amout is equal to actual parallel threads of CPU it has very good performance.


	\end{itemize}

	\newpage


	\section{References}

	\begin{itemize}
	  \item https://www.tug.org/twg/mactex/tutorials/ltxprimer-1.0.pdf.
	  \item https://gist.github.com/LeCoupa/122b12050f5fb267e75f
	  \item https://tex.stackexchange.com/questions
	  \item https://computing.llnl.gov/tutorials/pthreads
	  \item https://randu.org/tutorials/threads
	\end{itemize}

\end{document}

\grid
